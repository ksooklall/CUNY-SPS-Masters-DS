% Options for packages loaded elsewhere
\PassOptionsToPackage{unicode}{hyperref}
\PassOptionsToPackage{hyphens}{url}
%
\documentclass[
]{article}
\usepackage{lmodern}
\usepackage{amssymb,amsmath}
\usepackage{ifxetex,ifluatex}
\ifnum 0\ifxetex 1\fi\ifluatex 1\fi=0 % if pdftex
  \usepackage[T1]{fontenc}
  \usepackage[utf8]{inputenc}
  \usepackage{textcomp} % provide euro and other symbols
\else % if luatex or xetex
  \usepackage{unicode-math}
  \defaultfontfeatures{Scale=MatchLowercase}
  \defaultfontfeatures[\rmfamily]{Ligatures=TeX,Scale=1}
\fi
% Use upquote if available, for straight quotes in verbatim environments
\IfFileExists{upquote.sty}{\usepackage{upquote}}{}
\IfFileExists{microtype.sty}{% use microtype if available
  \usepackage[]{microtype}
  \UseMicrotypeSet[protrusion]{basicmath} % disable protrusion for tt fonts
}{}
\makeatletter
\@ifundefined{KOMAClassName}{% if non-KOMA class
  \IfFileExists{parskip.sty}{%
    \usepackage{parskip}
  }{% else
    \setlength{\parindent}{0pt}
    \setlength{\parskip}{6pt plus 2pt minus 1pt}}
}{% if KOMA class
  \KOMAoptions{parskip=half}}
\makeatother
\usepackage{xcolor}
\IfFileExists{xurl.sty}{\usepackage{xurl}}{} % add URL line breaks if available
\IfFileExists{bookmark.sty}{\usepackage{bookmark}}{\usepackage{hyperref}}
\hypersetup{
  pdftitle={Homework 8 DATA-605},
  pdfauthor={Kenan Sooklall},
  hidelinks,
  pdfcreator={LaTeX via pandoc}}
\urlstyle{same} % disable monospaced font for URLs
\usepackage[margin=1in]{geometry}
\usepackage{color}
\usepackage{fancyvrb}
\newcommand{\VerbBar}{|}
\newcommand{\VERB}{\Verb[commandchars=\\\{\}]}
\DefineVerbatimEnvironment{Highlighting}{Verbatim}{commandchars=\\\{\}}
% Add ',fontsize=\small' for more characters per line
\usepackage{framed}
\definecolor{shadecolor}{RGB}{248,248,248}
\newenvironment{Shaded}{\begin{snugshade}}{\end{snugshade}}
\newcommand{\AlertTok}[1]{\textcolor[rgb]{0.94,0.16,0.16}{#1}}
\newcommand{\AnnotationTok}[1]{\textcolor[rgb]{0.56,0.35,0.01}{\textbf{\textit{#1}}}}
\newcommand{\AttributeTok}[1]{\textcolor[rgb]{0.77,0.63,0.00}{#1}}
\newcommand{\BaseNTok}[1]{\textcolor[rgb]{0.00,0.00,0.81}{#1}}
\newcommand{\BuiltInTok}[1]{#1}
\newcommand{\CharTok}[1]{\textcolor[rgb]{0.31,0.60,0.02}{#1}}
\newcommand{\CommentTok}[1]{\textcolor[rgb]{0.56,0.35,0.01}{\textit{#1}}}
\newcommand{\CommentVarTok}[1]{\textcolor[rgb]{0.56,0.35,0.01}{\textbf{\textit{#1}}}}
\newcommand{\ConstantTok}[1]{\textcolor[rgb]{0.00,0.00,0.00}{#1}}
\newcommand{\ControlFlowTok}[1]{\textcolor[rgb]{0.13,0.29,0.53}{\textbf{#1}}}
\newcommand{\DataTypeTok}[1]{\textcolor[rgb]{0.13,0.29,0.53}{#1}}
\newcommand{\DecValTok}[1]{\textcolor[rgb]{0.00,0.00,0.81}{#1}}
\newcommand{\DocumentationTok}[1]{\textcolor[rgb]{0.56,0.35,0.01}{\textbf{\textit{#1}}}}
\newcommand{\ErrorTok}[1]{\textcolor[rgb]{0.64,0.00,0.00}{\textbf{#1}}}
\newcommand{\ExtensionTok}[1]{#1}
\newcommand{\FloatTok}[1]{\textcolor[rgb]{0.00,0.00,0.81}{#1}}
\newcommand{\FunctionTok}[1]{\textcolor[rgb]{0.00,0.00,0.00}{#1}}
\newcommand{\ImportTok}[1]{#1}
\newcommand{\InformationTok}[1]{\textcolor[rgb]{0.56,0.35,0.01}{\textbf{\textit{#1}}}}
\newcommand{\KeywordTok}[1]{\textcolor[rgb]{0.13,0.29,0.53}{\textbf{#1}}}
\newcommand{\NormalTok}[1]{#1}
\newcommand{\OperatorTok}[1]{\textcolor[rgb]{0.81,0.36,0.00}{\textbf{#1}}}
\newcommand{\OtherTok}[1]{\textcolor[rgb]{0.56,0.35,0.01}{#1}}
\newcommand{\PreprocessorTok}[1]{\textcolor[rgb]{0.56,0.35,0.01}{\textit{#1}}}
\newcommand{\RegionMarkerTok}[1]{#1}
\newcommand{\SpecialCharTok}[1]{\textcolor[rgb]{0.00,0.00,0.00}{#1}}
\newcommand{\SpecialStringTok}[1]{\textcolor[rgb]{0.31,0.60,0.02}{#1}}
\newcommand{\StringTok}[1]{\textcolor[rgb]{0.31,0.60,0.02}{#1}}
\newcommand{\VariableTok}[1]{\textcolor[rgb]{0.00,0.00,0.00}{#1}}
\newcommand{\VerbatimStringTok}[1]{\textcolor[rgb]{0.31,0.60,0.02}{#1}}
\newcommand{\WarningTok}[1]{\textcolor[rgb]{0.56,0.35,0.01}{\textbf{\textit{#1}}}}
\usepackage{graphicx,grffile}
\makeatletter
\def\maxwidth{\ifdim\Gin@nat@width>\linewidth\linewidth\else\Gin@nat@width\fi}
\def\maxheight{\ifdim\Gin@nat@height>\textheight\textheight\else\Gin@nat@height\fi}
\makeatother
% Scale images if necessary, so that they will not overflow the page
% margins by default, and it is still possible to overwrite the defaults
% using explicit options in \includegraphics[width, height, ...]{}
\setkeys{Gin}{width=\maxwidth,height=\maxheight,keepaspectratio}
% Set default figure placement to htbp
\makeatletter
\def\fps@figure{htbp}
\makeatother
\setlength{\emergencystretch}{3em} % prevent overfull lines
\providecommand{\tightlist}{%
  \setlength{\itemsep}{0pt}\setlength{\parskip}{0pt}}
\setcounter{secnumdepth}{-\maxdimen} % remove section numbering

\title{Homework 8 DATA-605}
\author{Kenan Sooklall}
\date{3/15/2021}

\begin{document}
\maketitle

\hypertarget{exercise-11}{%
\paragraph{Exercise 11}\label{exercise-11}}

A company buys 100 lightbulbs, each of which has an exponential lifetime
of 1000 hours. What is the expected time for the first of these bulbs to
burn out? (See Exercise 10.)

\begin{Shaded}
\begin{Highlighting}[]
\NormalTok{n=}\DecValTok{100}
\NormalTok{lifetime=}\DecValTok{1000}
\NormalTok{lifetime}\OperatorTok{/}\NormalTok{n}
\end{Highlighting}
\end{Shaded}

\begin{verbatim}
## [1] 10
\end{verbatim}

\hypertarget{exercise-14}{%
\paragraph{Exercise 14}\label{exercise-14}}

Assume that X\_1 and X\_2 are independent random variables, each having
an exponential density with parameter \(\lambda\). Show that
\(Z = X 1 − X 2\) has density

\(f_Z(z) = (1/2)λe^{−λ|z|}\)

Expanding the solution

\[f_Z(z) = λ/2e^{−λ|z|} \; \begin{cases} 
      λ/2e^{−λz} & z >= 0 \\
      λ/2e^{λz} & z < 0 \\
   \end{cases}\]

let \(X = X_1 \& Y = X_2 \rightarrow Z=X-Y \rightarrow Y = X - Z\)

For \(z >= 0 \rightarrow x \in (z , \infty)\)

\[ \int_{z}^{\infty}\lambda e ^{-\lambda x} \lambda e ^ {-\lambda(x-z)}dx = \lambda^2\int_{z}^{\infty} e ^{\lambda(z - 2x)} dx \]
Using u substitution

\(u = \lambda * (z - 2x) \rightarrow du=-2\lambda dx\)

Simplify
\[(-\frac{\lambda}{2}e^ {z - 2x}|_z^{\infty}) = \frac{\lambda}{2}e^{-\lambda z}\]
For \(z < 0 \rightarrow x \in (0 , \infty)\)

\[(-\frac{\lambda}{2}e^ {z - 2x}|_0^{\infty}) = \frac{\lambda}{2}e^{\lambda z}\]

\hypertarget{exercise-1}{%
\paragraph{Exercise 1}\label{exercise-1}}

Let X be a continuous random variable with mean \(\mu = 10\) and
variance \(σ^2 = 100/3\). Using Chebyshev's Inequality, find an upper
bound for the following probabilities

\(\sigma = 5.77\)

\begin{enumerate}
\def\labelenumi{(\alph{enumi})}
\item
  \(P (|X − 10| ≥ 2) \rightarrow 2 = k\sigma \rightarrow k=0.346 \rightarrow 1/k^2 = 0.12\)
\item
  P (\textbar X − 10\textbar{} ≥ 5)
\end{enumerate}

\begin{Shaded}
\begin{Highlighting}[]
\NormalTok{std =}\StringTok{ }\KeywordTok{sqrt}\NormalTok{(}\DecValTok{100}\OperatorTok{/}\DecValTok{3}\NormalTok{)}
\NormalTok{k =}\StringTok{ }\DecValTok{5}\OperatorTok{/}\NormalTok{std}
\DecValTok{1}\OperatorTok{/}\NormalTok{k}\OperatorTok{^}\DecValTok{2}
\end{Highlighting}
\end{Shaded}

\begin{verbatim}
## [1] 1.333333
\end{verbatim}

\begin{enumerate}
\def\labelenumi{(\alph{enumi})}
\setcounter{enumi}{2}
\tightlist
\item
  P (\textbar X − 10\textbar{} ≥ 9).
\end{enumerate}

\begin{Shaded}
\begin{Highlighting}[]
\NormalTok{k =}\StringTok{ }\DecValTok{9}\OperatorTok{/}\NormalTok{std}
\DecValTok{1}\OperatorTok{/}\NormalTok{k}\OperatorTok{^}\DecValTok{2}
\end{Highlighting}
\end{Shaded}

\begin{verbatim}
## [1] 0.4115226
\end{verbatim}

\begin{enumerate}
\def\labelenumi{(\alph{enumi})}
\setcounter{enumi}{3}
\tightlist
\item
  P (\textbar X − 10\textbar{} ≥ 20).
\end{enumerate}

\begin{Shaded}
\begin{Highlighting}[]
\NormalTok{k =}\StringTok{ }\DecValTok{20}\OperatorTok{/}\NormalTok{std}
\DecValTok{1}\OperatorTok{/}\NormalTok{k}\OperatorTok{^}\DecValTok{2}
\end{Highlighting}
\end{Shaded}

\begin{verbatim}
## [1] 0.08333333
\end{verbatim}

\end{document}
