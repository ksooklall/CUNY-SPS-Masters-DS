\documentclass[5p]{elsarticle} %review=doublespace preprint=single 5p=2 column
%%% Begin My package additions %%%%%%%%%%%%%%%%%%%

\usepackage[hyphens]{url}


\usepackage{lineno} % add

\usepackage{graphicx}
%%%%%%%%%%%%%%%% end my additions to header

\usepackage[T1]{fontenc}
\usepackage{lmodern}
\usepackage{amssymb,amsmath}
\usepackage{ifxetex,ifluatex}
\usepackage{fixltx2e} % provides \textsubscript
% use upquote if available, for straight quotes in verbatim environments
\IfFileExists{upquote.sty}{\usepackage{upquote}}{}
\ifnum 0\ifxetex 1\fi\ifluatex 1\fi=0 % if pdftex
  \usepackage[utf8]{inputenc}
\else % if luatex or xelatex
  \usepackage{fontspec}
  \ifxetex
    \usepackage{xltxtra,xunicode}
  \fi
  \defaultfontfeatures{Mapping=tex-text,Scale=MatchLowercase}
  \newcommand{\euro}{€}
\fi
% use microtype if available
\IfFileExists{microtype.sty}{\usepackage{microtype}}{}
\usepackage[]{natbib}
\bibliographystyle{plainnat}

\usepackage{graphicx}
\ifxetex
  \usepackage[setpagesize=false, % page size defined by xetex
              unicode=false, % unicode breaks when used with xetex
              xetex]{hyperref}
\else
  \usepackage[unicode=true]{hyperref}
\fi
\hypersetup{breaklinks=true,
            bookmarks=true,
            pdfauthor={},
            pdftitle={Readmission rates fo Type 2 Diabetics},
            colorlinks=false,
            urlcolor=blue,
            linkcolor=magenta,
            pdfborder={0 0 0}}

\setcounter{secnumdepth}{0}
% Pandoc toggle for numbering sections (defaults to be off)
\setcounter{secnumdepth}{0}


% tightlist command for lists without linebreak
\providecommand{\tightlist}{%
  \setlength{\itemsep}{0pt}\setlength{\parskip}{0pt}}



\usepackage{booktabs}
\usepackage{longtable}
\usepackage{array}
\usepackage{multirow}
\usepackage{wrapfig}
\usepackage{float}
\usepackage{colortbl}
\usepackage{pdflscape}
\usepackage{tabu}
\usepackage{threeparttable}
\usepackage{threeparttablex}
\usepackage[normalem]{ulem}
\usepackage{makecell}
\usepackage{xcolor}



\begin{document}


\begin{frontmatter}

  \title{Readmission rates fo Type 2 Diabetics}
    \author[CUNY-SPS]{Kenan Sooklall%
  %
  \fnref{Corresponding Author}}
   \ead{ksooklall100@gmail.com} 
      \affiliation[]{CUNY School of Professional Studies}
    \cortext[cor1]{Corresponding author}
  
  \begin{abstract}
  This is the abstract.

  It consists of two paragraphs.
  \end{abstract}
  
 \end{frontmatter}

\hypertarget{introduction}{%
\subsubsection{Introduction}\label{introduction}}

Type 2 diabetes has reached epidemic proportions worldwide. According to
the WHO diabetes has entered the top 10 causes of death, following a
significant percentage increase of 70\% since 2000. Diabetes is also
responsible for the largest rise in male deaths among the top 10, with
an 80\% increase since 2000. Diabetes is a group of diseases that affect
insulin production and use. In type 1 diabetes, the pancreas can't
produce insulin. The cause isn't known. In type 2 diabetes, the pancreas
doesn't produce enough insulin, or insulin can't be used effectively.
Type 2 diabetes can be caused by a number of factors, including poor
diet, lack of exercise, and being overweight.

\hypertarget{data-analysis}{%
\subsubsection{Data Analysis}\label{data-analysis}}

The dataset was obtained from UC Irvine and contains 101766 rows and 50
columns. The patient column contains 71518 unique patients. To avoid
overfitting on a single patient all duplicated patient data was dropped
leaving the dataset with 71518 rows. The 50 columns can be broken down
in to 5 parts, admission data, lab diagnosis, patient details, diagnosis
data and diabetic drugs.

\clearpage
\onecolumn

\begin{table}
\centering
\begin{tabular}{|>{\raggedright\arraybackslash}p{9em}|>{}l|>{\raggedright\arraybackslash}p{35em}|>{\raggedleft\arraybackslash}p{1em}}
\hline
Feature.Name & Type & Description.and.Values & missing\\
\hline
\textbf{Encounter ID} & \cellcolor{yellow}{Numeric} & Unique identifier of an encounter & 0.0\\
\hline
\textbf{Patient number} & \cellcolor{yellow}{Numeric} & Unique identifier of a patient & 0.0\\
\hline
\textbf{Race} & \cellcolor{yellow}{Nominal} & Caucasian, Asian, African American, Hispanic, and other & 2.2\\
\hline
\textbf{Gender Values} & \cellcolor{yellow}{Nominal} & male, female, and unknown/invalid & 0.0\\
\hline
\textbf{Age} & \cellcolor{yellow}{Nominal} & Grouped in 10-year intervals 0, 10), 10, 20), …, 90, 100) & 0.0\\
\hline
\textbf{Weight} & \cellcolor{yellow}{Numeric} & Weight in pounds & 96.9\\
\hline
\textbf{Admission type} & \cellcolor{yellow}{Nominal} & Integer identifier corresponding to 9 distinct values, for example, emergency, urgent, elective, newborn, and not available & 0.0\\
\hline
\textbf{Discharge disposition} & \cellcolor{yellow}{Nominal} & Integer identifier corresponding to 29 distinct values, for example, discharged to home, expired, and not available & 0.0\\
\hline
\textbf{Admission source} & \cellcolor{yellow}{Nominal} & Integer identifier corresponding to 21 distinct values, for example, physician referral, emergency room, and transfer from a hospital & 0.0\\
\hline
\textbf{Time in hospital} & \cellcolor{yellow}{Numeric} & Integer number of days between admission and discharge & 0.0\\
\hline
\textbf{Payer code} & \cellcolor{yellow}{Nominal} & Integer identifier corresponding to 23 distinct values, for example, Blue Cross/Blue Shield, Medicare, and self-pay Medical & 39.6\\
\hline
\textbf{Medical specialty} & \cellcolor{yellow}{Nominal} & Integer identifier of a specialty of the admitting physician, corresponding to 84 distinct values, for example, cardiology, internal medicine, family/general practice, and surgeon & 49.1\\
\hline
\textbf{Number of lab procedures} & \cellcolor{yellow}{Numeric} & Number of lab tests performed during the encounter & 0.0\\
\hline
\textbf{Number of procedures} & \cellcolor{yellow}{Numeric} & Number of procedures (other than lab tests) performed during the encounter & 0.0\\
\hline
\textbf{Number of medications} & \cellcolor{yellow}{Numeric} & Number of distinct generic names administered during the encounter & 0.0\\
\hline
\textbf{Number of outpatient visits} & \cellcolor{yellow}{Numeric} & Number of outpatient visits of the patient in the year preceding the encounter & 0.0\\
\hline
\textbf{Number of emergency visits} & \cellcolor{yellow}{Numeric} & Number of emergency visits of the patient in the year preceding the encounter & 0.0\\
\hline
\textbf{Number of inpatient visits} & \cellcolor{yellow}{Numeric} & Number of inpatient visits of the patient in the year preceding the encounter & 0.0\\
\hline
\textbf{Diagnosis 1} & \cellcolor{yellow}{Nominal} & The primary diagnosis (coded as first three digits of ICD9); 848 distinct values & 0.0\\
\hline
\textbf{Diagnosis 2} & \cellcolor{yellow}{Nominal} & Secondary diagnosis (coded as first three digits of ICD9); 923 distinct values & 0.4\\
\hline
\textbf{Diagnosis 3} & \cellcolor{yellow}{Nominal} & Additional secondary diagnosis (coded as first three digits of ICD9); 954 distinct values & 1.4\\
\hline
\textbf{Number of diagnoses} & \cellcolor{yellow}{Numeric} & Number of diagnoses entered to the system 0\% & 0.0\\
\hline
\textbf{Glucose serum test} & \cellcolor{yellow}{Nominal} & result Indicates the range of the result or if the test was not taken. Values '>200', '>300', 'normal' and 'none' if not measured & 0.0\\
\hline
\textbf{A1c test result} & \cellcolor{yellow}{Nominal} & Indicates the range of the result or if the test was not taken. Values '>8' if the result was greater than 8\%, '>7' if the result was greater than 7\% but less than 8\%, 'normal' if the result was less than 7\%, and “none” if not measured. 7\% to 8\% is an increase of \textasciitilde{}30mg/dl of glucose, 6\%=120mg/dl & 0.0\\
\hline
\textbf{Change of medications} & \cellcolor{yellow}{Nominal} & Indicates if there was a change in diabetic medications (either dosage or generic name). Values 'change' and 'no change' & 0.0\\
\hline
\textbf{Diabetes medications} & \cellcolor{yellow}{Nominal} & Indicates if there was any diabetic medication prescribed. Values 'yes' and 'no' & 0.0\\
\hline
\textbf{24 features for medications} & \cellcolor{yellow}{Nominal} & glyburide-metformin, glipizide-metformin, glimepiride-pioglitazone, metformin-rosiglitazone, and metformin-pioglitazone, the feature indicates whether the drug was prescribed or there was a change in the dosage. Values 'up' if the dosage was increased during the encounter, 'down' if the dosage was decreased, 'steady' if the dosage did not change, and 'no' if the drug was not prescribed & 0.0\\
\hline
\textbf{Readmitted Days} & \cellcolor{yellow}{Nominal} & Values '<30' if the patient was readmitted in less than 30 days, '>30' if the patient was readmitted in more than 30 days, and 'No' for no record of readmission & 0.0\\
\hline
\end{tabular}
\end{table}

\clearpage
\twocolumn

\begin{figure}
\hypertarget{id}{%
\centering
\includegraphics[width=3.64583in,height=2.08333in]{/home/kenan/Documents/learning/masters/CUNY-SPS-Masters-DS/DATA_698/graphs/age.jpg}
\caption{Age vs Readmission}\label{id}
}
\end{figure}

\hypertarget{modeling}{%
\subsubsection{Modeling}\label{modeling}}

\hypertarget{results-and-dicussion}{%
\subsubsection{Results and Dicussion}\label{results-and-dicussion}}

\hypertarget{conclusion}{%
\subsubsection{Conclusion}\label{conclusion}}


\end{document}
