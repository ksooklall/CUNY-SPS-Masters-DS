\documentclass[5p]{elsarticle} %review=doublespace preprint=single 5p=2 column
%%% Begin My package additions %%%%%%%%%%%%%%%%%%%

\usepackage[hyphens]{url}


\usepackage{lineno} % add

\usepackage{graphicx}
%%%%%%%%%%%%%%%% end my additions to header

\usepackage[T1]{fontenc}
\usepackage{lmodern}
\usepackage{amssymb,amsmath}
\usepackage{ifxetex,ifluatex}
\usepackage{fixltx2e} % provides \textsubscript
% use upquote if available, for straight quotes in verbatim environments
\IfFileExists{upquote.sty}{\usepackage{upquote}}{}
\ifnum 0\ifxetex 1\fi\ifluatex 1\fi=0 % if pdftex
  \usepackage[utf8]{inputenc}
\else % if luatex or xelatex
  \usepackage{fontspec}
  \ifxetex
    \usepackage{xltxtra,xunicode}
  \fi
  \defaultfontfeatures{Mapping=tex-text,Scale=MatchLowercase}
  \newcommand{\euro}{€}
\fi
% use microtype if available
\IfFileExists{microtype.sty}{\usepackage{microtype}}{}
\usepackage[]{natbib}
\bibliographystyle{plainnat}

\usepackage{graphicx}
\ifxetex
  \usepackage[setpagesize=false, % page size defined by xetex
              unicode=false, % unicode breaks when used with xetex
              xetex]{hyperref}
\else
  \usepackage[unicode=true]{hyperref}
\fi
\hypersetup{breaklinks=true,
            bookmarks=true,
            pdfauthor={},
            pdftitle={Analysis of readmission rates for Type 2 Diabetics},
            colorlinks=false,
            urlcolor=blue,
            linkcolor=magenta,
            pdfborder={0 0 0}}

\setcounter{secnumdepth}{0}
% Pandoc toggle for numbering sections (defaults to be off)
\setcounter{secnumdepth}{0}


% tightlist command for lists without linebreak
\providecommand{\tightlist}{%
  \setlength{\itemsep}{0pt}\setlength{\parskip}{0pt}}



\usepackage{booktabs}
\usepackage{longtable}
\usepackage{array}
\usepackage{multirow}
\usepackage{wrapfig}
\usepackage{float}
\usepackage{colortbl}
\usepackage{pdflscape}
\usepackage{tabu}
\usepackage{threeparttable}
\usepackage{threeparttablex}
\usepackage[normalem]{ulem}
\usepackage{makecell}
\usepackage{xcolor}



\begin{document}


\begin{frontmatter}

  \title{Analysis of readmission rates for Type 2 Diabetics}
    \author[CUNY-SPS]{Kenan Sooklall}
   \ead{ksooklall100@gmail.com} 
      \affiliation[]{CUNY School of Professional Studies}
    \cortext[cor1]{Corresponding author}
  
  \begin{abstract}
  Type 2 diabetes is a disease that affects many around the world and
  continues to cause wide ranging problems. In this paper four models
  with various combinations of features that involve patient admission
  id, diabetic drug use, lab data and others were trained to predict the
  probability of a patient being readmitted. The results enforce the
  difficulties of making predictions with patients affected by type 2
  diabetes. All models have an F1-score of \textasciitilde55\% while the
  model three has an F1-score of to 78\% only for the positive label.
  For all models age always played a large role in predicting
  readmission. Diabetic drugs glipizide and rosiglitazone were
  determined to be important in a patient being readmitted. The initial
  cause of the hospital visit, was determined to be the most important
  variable when using all features.
  \end{abstract}
  
 \end{frontmatter}

\hypertarget{introduction}{%
\subsubsection{Introduction}\label{introduction}}

Type 2 diabetes has reached epidemic proportions worldwide. According to
the WHO diabetes has entered the top 10 causes of death, following a
significant percentage increase of 70\% since 2000. Diabetes is a group
of diseases that affect insulin production and use. In type 1 diabetes,
the pancreas can't produce insulin. In type 2 diabetes, the pancreas
doesn't produce enough insulin, or insulin can't be used effectively.
Type 2 diabetes can be caused by a number of factors, including poor
diet, lack of exercise, and being overweight.

Throughout this paper diabetes is a shorthand for type 2 diabetes.
According to the source the dataset represents 10 years (1999-2008) of
clinical care at 130 US hospitals and integrated delivery networks. It
includes over 50 features representing patient and hospital outcomes.
The data contained information that satisfied the following criteria. It
is an inpatient encounter (a hospital admission). It is a diabetic
encounter, that is, one during which any kind of diabetes was entered to
the system as a diagnosis. The length of stay was at least 1 day and at
most 14 days. Laboratory tests were performed during the encounter.
Medications were administered during the encounter.

\hypertarget{preprocessing}{%
\subsubsection{Preprocessing}\label{preprocessing}}

The data set was obtained from UC Irvine and contains 101766 rows and 50
columns. The patient column contains 71518 unique patients with 28784
readmitted and 28781 not readmitted. To avoid over fitting on a single
patient all duplicated patient data was dropped. Then down sampling was
conducted for the minority class leaving the data set with 57568 rows.
The 50 columns can be broken down in to 5 parts, admission data, lab
diagnosis, patient details, diagnosis data and diabetic drugs, Table 1.

All ``?'' signify a missing value, so ``?'' were replaced with NaNs for
all columns. The weight, payer code and medical specialty were dropped
for having too many NaNs. Drug columns examide, citoglipton and
glimepiride-pioglitazone were dropped because they had 1 unique value.
Encounter ID was dropped as it doesn't provide any informative
information. Race was dropped as it might make the model not useful and
can lead to uninformative result.

The target column readmitted was transformed with NO as 0 and all other
values as 1. This remapping transforms readmitted from 3 unique values
to 2, changing this analysis to binary classification. The age column
was bucketed into young (age\textless30), adult
(30\textless age\textless60) and old (age\textgreater60) then one hot
encoded.

The diagnosis columns diag\_1,2 and 3 were mapped according to the icd9
codes, then one hot encoded. The drug columns were mapped NO as 0 and
all other values as 1. The only NaN rows were diagnosis, so they were
all dropped as imputation is not possible. Admission type ID, Discharge
disposition ID and Admission source ID were mapping according to the
source paper with a few major changes. Discharge disposition ID was
reduced to ``other'' , ``home'' and NaN. Admission source ID was reduced
to ``emergency'', ``referral'', ``other'' and NaN. Admission type ID was
not changed from the source except replacing NULL and not available with
NaN.

Columns to great interest are glucose serum test result and A1c test
result. Glucose serum test is a numeric value that measures blood sugar
level, a value of 126 mg/dL or higher indicates the patient is diabetic.
The A1C test measures the percentage of your red blood cells that have
sugar-coated hemoglobin, a value of 6.5\% or above indicates the patient
is diabetic.

\clearpage
\onecolumn

\begin{table}
\centering
\begin{tabular}{|>{\raggedright\arraybackslash}p{9em}|>{}l|>{\raggedright\arraybackslash}p{35em}|>{\raggedleft\arraybackslash}p{1em}}
\hline
Feature.Name & Type & Description.and.Values & missing\\
\hline
\textbf{Encounter ID} & \cellcolor{yellow}{Numeric} & Unique identifier of an encounter & 0.0\\
\hline
\textbf{Patient number} & \cellcolor{yellow}{Numeric} & Unique identifier of a patient & 0.0\\
\hline
\textbf{Race} & \cellcolor{yellow}{Nominal} & Caucasian, Asian, African American, Hispanic, and other & 2.2\\
\hline
\textbf{Gender Values} & \cellcolor{yellow}{Nominal} & male, female, and unknown/invalid & 0.0\\
\hline
\textbf{Age} & \cellcolor{yellow}{Nominal} & Grouped in 10-year intervals 0, 10), 10, 20), …, 90, 100) & 0.0\\
\hline
\textbf{Weight} & \cellcolor{yellow}{Numeric} & Weight in pounds & 96.9\\
\hline
\textbf{Admission type} & \cellcolor{yellow}{Nominal} & Integer identifier corresponding to 9 distinct values, for example, emergency, urgent, elective, newborn, and not available & 0.0\\
\hline
\textbf{Discharge disposition} & \cellcolor{yellow}{Nominal} & Integer identifier corresponding to 29 distinct values, for example, discharged to home, expired, and not available & 0.0\\
\hline
\textbf{Admission source} & \cellcolor{yellow}{Nominal} & Integer identifier corresponding to 21 distinct values, for example, physician referral, emergency room, and transfer from a hospital & 0.0\\
\hline
\textbf{Time in hospital} & \cellcolor{yellow}{Numeric} & Integer number of days between admission and discharge & 0.0\\
\hline
\textbf{Payer code} & \cellcolor{yellow}{Nominal} & Integer identifier corresponding to 23 distinct values, for example, Blue Cross/Blue Shield, Medicare, and self-pay Medical & 39.6\\
\hline
\textbf{Medical specialty} & \cellcolor{yellow}{Nominal} & Integer identifier of a specialty of the admitting physician, corresponding to 84 distinct values, for example, cardiology, internal medicine, family/general practice, and surgeon & 49.1\\
\hline
\textbf{Number of lab procedures} & \cellcolor{yellow}{Numeric} & Number of lab tests performed during the encounter & 0.0\\
\hline
\textbf{Number of procedures} & \cellcolor{yellow}{Numeric} & Number of procedures (other than lab tests) performed during the encounter & 0.0\\
\hline
\textbf{Number of medications} & \cellcolor{yellow}{Numeric} & Number of distinct generic names administered during the encounter & 0.0\\
\hline
\textbf{Number of outpatient visits} & \cellcolor{yellow}{Numeric} & Number of outpatient visits of the patient in the year preceding the encounter & 0.0\\
\hline
\textbf{Number of emergency visits} & \cellcolor{yellow}{Numeric} & Number of emergency visits of the patient in the year preceding the encounter & 0.0\\
\hline
\textbf{Number of inpatient visits} & \cellcolor{yellow}{Numeric} & Number of inpatient visits of the patient in the year preceding the encounter & 0.0\\
\hline
\textbf{Diagnosis 1} & \cellcolor{yellow}{Nominal} & The primary diagnosis (coded as first three digits of ICD9); 848 distinct values & 0.0\\
\hline
\textbf{Diagnosis 2} & \cellcolor{yellow}{Nominal} & Secondary diagnosis (coded as first three digits of ICD9); 923 distinct values & 0.4\\
\hline
\textbf{Diagnosis 3} & \cellcolor{yellow}{Nominal} & Additional secondary diagnosis (coded as first three digits of ICD9); 954 distinct values & 1.4\\
\hline
\textbf{Number of diagnoses} & \cellcolor{yellow}{Numeric} & Number of diagnoses entered to the system 0\% & 0.0\\
\hline
\textbf{Glucose serum test} & \cellcolor{yellow}{Nominal} & result Indicates the range of the result or if the test was not taken. Values '>200', '>300', 'normal' and 'none' if not measured & 0.0\\
\hline
\textbf{A1c test result} & \cellcolor{yellow}{Nominal} & Indicates the range of the result or if the test was not taken. Values '>8' if the result was greater than 8\%, '>7' if the result was greater than 7\% but less than 8\%, 'normal' if the result was less than 7\%, and “none” if not measured. 7\% to 8\% is an increase of \textasciitilde{}30mg/dl of glucose, 6\%=120mg/dl & 0.0\\
\hline
\textbf{Change of medications} & \cellcolor{yellow}{Nominal} & Indicates if there was a change in diabetic medications (either dosage or generic name). Values 'change' and 'no change' & 0.0\\
\hline
\textbf{Diabetes medications} & \cellcolor{yellow}{Nominal} & Indicates if there was any diabetic medication prescribed. Values 'yes' and 'no' & 0.0\\
\hline
\textbf{24 features for medications} & \cellcolor{yellow}{Nominal} & glyburide-metformin, glipizide-metformin, glimepiride-pioglitazone, metformin-rosiglitazone, and metformin-pioglitazone, the feature indicates whether the drug was prescribed or there was a change in the dosage. Values 'up' if the dosage was increased during the encounter, 'down' if the dosage was decreased, 'steady' if the dosage did not change, and 'no' if the drug was not prescribed & 0.0\\
\hline
\textbf{Readmitted Days} & \cellcolor{yellow}{Nominal} & Values '<30' if the patient was readmitted in less than 30 days, '>30' if the patient was readmitted in more than 30 days, and 'No' for no record of readmission & 0.0\\
\hline
\end{tabular}
\end{table}

\clearpage
\twocolumn

\hypertarget{exploratory-data-analysis}{%
\subsubsection{Exploratory Data
Analysis}\label{exploratory-data-analysis}}

The data set is slightly skewed with 42982 (60\%) patients not being
readmitted and 28533 (40\%) readmitted. Gender is well balanced at 53\%
male and 47\% female. Age is skewed more toward the older population,
which is expected for a disease that is more common in the older
population. All other numeric variables follow a similar distribution.

There is a weak relationship with drugs taken and a patient being
readmitted. The only drug with a high readmission is insulin which is
expected since insulin plays a major role in a patient becoming
diabetic. Figure 1 shows that most of the readmitting comes from older
individuals. The older population is represented the most and the
younger population the least. Figure 2 shows that A1C results greater
than 8\% have a big affect on readmission; however, for patients who
didn't get readmitted there are a lot of adults with A1C results of
greater than 8\%. Figure 3 shows a big spike for normal glucose levels
for both classes. There is also very similar levels of all age groups
between both classes. For both A1C results and max glucose serum the
younger portion of the population contributed the least to readmission.

Figure 4 shows how time in hospital is right skewed with no distinction
between classes. Figure 5 and 6 shows the number of lab procedures and
number of medications are normally distributed with the former having a
spike at the start and the ladder being slightly right skewed.

\begin{figure}
\hypertarget{id}{%
\centering
\includegraphics[width=3.125in,height=2.08333in]{/home/kenan/Documents/learning/masters/CUNY-SPS-Masters-DS/DATA_698/graphs/age.jpg}
\caption{Age vs Readmission}\label{id}
}
\end{figure}

\begin{figure}
\hypertarget{id}{%
\centering
\includegraphics[width=3.125in,height=2.08333in]{/home/kenan/Documents/learning/masters/CUNY-SPS-Masters-DS/DATA_698/graphs/A1Cresults.jpg}
\caption{A1Cresults vs readmission}\label{id}
}
\end{figure}

\begin{figure}
\hypertarget{id}{%
\centering
\includegraphics[width=3.125in,height=2.08333in]{/home/kenan/Documents/learning/masters/CUNY-SPS-Masters-DS/DATA_698/graphs/max_glu_serum.jpg}
\caption{Max Glucose Serum vs readmission}\label{id}
}
\end{figure}

\begin{figure}
\hypertarget{id}{%
\centering
\includegraphics[width=3.125in,height=2.08333in]{/home/kenan/Documents/learning/masters/CUNY-SPS-Masters-DS/DATA_698/graphs/time_in_hospital.jpg}
\caption{Time in hospital}\label{id}
}
\end{figure}

\begin{figure}
\hypertarget{id}{%
\centering
\includegraphics[width=3.125in,height=2.08333in]{/home/kenan/Documents/learning/masters/CUNY-SPS-Masters-DS/DATA_698/graphs/num_lab_procedures.jpg}
\caption{Number of lab procedures}\label{id}
}
\end{figure}

\begin{figure}
\hypertarget{id}{%
\centering
\includegraphics[width=3.125in,height=2.08333in]{/home/kenan/Documents/learning/masters/CUNY-SPS-Masters-DS/DATA_698/graphs/num_medications.jpg}
\caption{Number of medications}\label{id}
}
\end{figure}

\hypertarget{modeling}{%
\subsubsection{Modeling}\label{modeling}}

Three models SVM, Logistic Regression and Random Forest Classifier were
trained with cross validation using different combinations of features
and tuned with grid search. The dataset was split into training and
testing with the test set containing 15\% of the data. Numeric columns
were standardized by subtracting the mean and dividing by the standard
deviation thus having a mean of 0 and std of 1. All other columns were
fed into the model without further processing unless specified. After
trying many combinations of features 4 specific sets stood out.

The first model is trained on a using all features, lab data, patient
data except drug data. Max glu serum and A1C result were engineered to
become binary with 0 as normal and all other values as 1. The training
set contained 48931 patients and 8636 for testing. The second model
focused on A1C levels along with age, gender and insulin. All rows with
NaN values for A1C results were dropped and then the A1C column was one
hot encoded for the 3 unique cases. This model the training set
contained 8492 patients and 1499 for testing. The third model focused on
A1C levels and blood glucose levels. All row containing NaN for both A1C
and max glu serum were dropped leaving only only 207 rows for training
and testing. This very small dataset produced the most extreme results.
The final model focused on the drug columns. Since no rows were dropped
the sample size is the same as model 1 except it has 21 features.

\hypertarget{results}{%
\subsubsection{Results}\label{results}}

The results from cross validation produced very similar scores for all
architectures but random forest produced the best score most of the
time.

The results of the model 1 from Table 1 show higher recall for the
positive class and higher precision for the negative class. The high
recall states the model is good at catching false negative so if the
model predicts that you won't get readmitted it's more likely to be
correct. The higher precision over recall for the negative class states
the model is good at catching false positive. Model 1 has higher
f1-score for the positive class. The feature importance in Figure 7
shows highest importance with the admission type id followed by number
of diagnoses. These feature importance makes sense since an admission
type id of ``emergency'' and having a lot of diagnoses intuitively
correlate to a patient being readmitted.

Table 2 and Figure 8 show the results of model 2. The importance of age
played a very strong role for this model, almost twice as important as
the importance of being young. A1C result and gender play a much smaller
role. The classification report doesn't show any metric that stand out.
The F1-score for the positive label is 6 points higher than the
negative.

The results for model 3 are expected when the sample size is so small.
From table 3 we see 0 for both precision and recall for the negative
label. If this model predicted a patient won't be readmitted it
shouldn't be trusted at all. Likewise this model would be accurate if it
predicts the patient will be readmitted. Figure 9 tells a strange story.
Glucose greater than 200 has the most importance but high A1C levels are
at the bottom with age being in the middle.

Finally model 4 has similar results as model 2 except recall is better
for the negative class. The drugs glipizide and rosiglitazone were the
most import according to Figure 10. Change being the second most
important does have some intuitive reasoning.

\clearpage
\onecolumn

\begin{table}

\caption{\label{tab:unnamed-chunk-3}Model 1 Classification Report}
\centering
\begin{tabular}[t]{l|r|r|r|r}
\hline
target & precision & recall & f1.score & support\\
\hline
no & 0.59 & 0.48 & 0.53 & 4377\\
\hline
yes & 0.55 & 0.66 & 0.60 & 4258\\
\hline
\end{tabular}
\end{table}

\begin{figure}
\hypertarget{id}{%
\centering
\includegraphics[width=6.25in,height=3.125in]{/home/kenan/Documents/learning/masters/CUNY-SPS-Masters-DS/DATA_698/graphs/m1_feature_importance.jpg}
\caption{Model 1 Feature Importance}\label{id}
}
\end{figure}

\begin{table}

\caption{\label{tab:unnamed-chunk-4}Model 2 Classification Report}
\centering
\begin{tabular}[t]{l|r|r|r|r}
\hline
target & precision & recall & f1.score & support\\
\hline
no & 0.55 & 0.47 & 0.51 & 763\\
\hline
yes & 0.53 & 0.61 & 0.57 & 745\\
\hline
\end{tabular}
\end{table}

\begin{figure}
\hypertarget{id}{%
\centering
\includegraphics[width=6.25in,height=3.125in]{/home/kenan/Documents/learning/masters/CUNY-SPS-Masters-DS/DATA_698/graphs/m2_feature_importance.jpg}
\caption{Model 2 Feature Importance}\label{id}
}
\end{figure}

\begin{table}

\caption{\label{tab:unnamed-chunk-5}Model 3 Classification Report}
\centering
\begin{tabular}[t]{l|r|r|r|r}
\hline
target & precision & recall & f1.score & support\\
\hline
no & 0.00 & 0 & 0.00 & 11\\
\hline
yes & 0.65 & 1 & 0.78 & 20\\
\hline
\end{tabular}
\end{table}

\begin{figure}
\hypertarget{id}{%
\centering
\includegraphics[width=6.25in,height=3.125in]{/home/kenan/Documents/learning/masters/CUNY-SPS-Masters-DS/DATA_698/graphs/m3_feature_importance.jpg}
\caption{Model 3 Feature Importance}\label{id}
}
\end{figure}

\begin{table}

\caption{\label{tab:unnamed-chunk-6}Model 4 Classification Report}
\centering
\begin{tabular}[t]{l|r|r|r|r}
\hline
target & precision & recall & f1.score & support\\
\hline
no & 0.53 & 0.59 & 0.56 & 4335\\
\hline
yes & 0.53 & 0.47 & 0.50 & 4300\\
\hline
\end{tabular}
\end{table}

\includegraphics[width=6.25in,height=3.125in]{/home/kenan/Documents/learning/masters/CUNY-SPS-Masters-DS/DATA_698/graphs/m4_feature_importance.jpg}
\clearpage \twocolumn

\hypertarget{dicussion}{%
\subsubsection{Dicussion}\label{dicussion}}

Diabetes is a very complex disease and this analysis has shown how
difficult it is to predict and manage once a patient is positive. The 4
models build from various combinations of features produced slight
better predictions than flipping a coin. Model 2 and 3 produced
non-intuitively results, we would expected very high blood glucose level
(\textgreater300) and very high A1C results (\textgreater8) to play the
strongest role; however they didn't. One explanation could be that the
patients passed a ways before being readmitted. Figure 1 showed that
most of the data set comes from the elderly population so it's very
possible. Having such high results over the age of 60 is more
detrimental to one health as opposed to an younger person.

Other features like time in hospital, num lab procedures, number
diagnoses and num medications didn't play as large a role as what we
would have expected; however the Figures 4-6 did propose that fact.
Furthermore the four models trained here are by no means the best use of
the dataset nor were they exhaustive. Along with different combinations
of features and feature engineering more results could be found.

\hypertarget{conclusion}{%
\subsubsection{Conclusion}\label{conclusion}}

According to the mayo clinic type 2 diabetes is disease in the way the
body regulates and uses glucose as a fuel. This long-term (chronic)
condition results in too much sugar circulating in the bloodstream.
Eventually, high blood sugar levels can lead to disorders of the
circulatory, nervous and immune systems. Although there are many
diabetic medications glipizide and rosiglitazone showed large importance
followed by insulin. However age seems to be the most important variable
for patients to be readmitted to the hospital.

\hypertarget{source-citation}{%
\subsubsection{Source \& Citation}\label{source-citation}}

The data are submitted on behalf of the Center for Clinical and
Translational Research, Virginia Commonwealth University, a recipient of
NIH CTSA grant UL1 TR00058 and a recipient of the CERNER data. John
Clore (jclore `@' vcu.edu), Krzysztof J. Cios (kcios `@' vcu.edu), Jon
DeShazo (jpdeshazo `@' vcu.edu), and Beata Strack (strackb `@' vcu.edu).
This data is a de-identified abstract of the Health Facts database
(Cerner Corporation, Kansas City, MO).

Beata Strack, Jonathan P. DeShazo, Chris Gennings, Juan L. Olmo,
Sebastian Ventura, Krzysztof J. Cios, and John N. Clore, ``Impact of
HbA1c Measurement on Hospital Readmission Rates: Analysis of 70,000
Clinical Database Patient Records,'' BioMed Research International,
vol.~2014, Article ID 781670, 11 pages, 2014.

\hypertarget{references}{%
\subsubsection{References}\label{references}}

https://www.cdc.gov/diabetes/basics/getting-tested

https://www.mayoclinic.org/diseases-conditions/type-2-diabetes/symptoms-causes/syc-20351193

https://archive-beta.ics.uci.edu/ml/datasets/diabetes+130+us+hospitals+for+years+1999+2008


\end{document}
